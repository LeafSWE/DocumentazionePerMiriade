\documentclass[../DocumentazioneDelloStudio.tex]{subfiles}

\begin{document}
\section{Scelte software}
	Di seguito sono riportate le scelte software fatte.

	\subsection{Android}
		È stato scelto di sviluppare un'applicazione per dispositivi mobile Android rispetto ad IOS per la disponibilità prevalente di dispositivi Android.
		\begin{itemize} 
			\item Versione minima 4.4 
			\item Versione massima 6.0
		\end{itemize}

	\subsection{SQLite}
		Libreria che implementa un DBMS SQL transazionale senza la necessità di un server. Viene utilizzata per salvare e gestire le mappe scaricate e installate nel dispositivo e il relativo contenuto.
		\begin{itemize} \item Versione utilizzata 3.9.2\end{itemize}

	\subsection{AltBeacon}
		Libreria che permette ai sistemi operativi mobile di interfacciarsi ai beacon, offrendo molteplici funzionalità. Viene utilizzata per permettere la comunicazione tra l'applicativo Android e i beacon.
		\begin{itemize} \item Versione utilizzata 2.02\end{itemize}
		
	\subsection{JGraphT}
		Libreria Java che fornisce funzionalità matematiche per modellare grafi. Viene utilizzata per la rappresentazione delle mappe e per il calcolo dei percorsi. 
		\begin{itemize} \item Versione utilizzata 0.9.1\end{itemize}

	\subsection{Gson}
		Libreria Java che fornisce funzionalità per la gestione di oggetti JSON. Tale libreria è utilizzata la gestione del download delle mappe da remoto. 
		\begin{itemize} \item Versione utilizzata 2.6.2\end{itemize}

	\subsection{Dagger}
		Libreria Android utilizzata per effettuare la dependency injection. Viene utilizzata per la creazione dei singleton.
		\begin{itemize} \item Versione utilizzata 2.0\end{itemize}

	\subsection{Picasso}
		Libreria per la gestione delle immagini in remoto. Viene utilizzata per scaricare le immagini utilizzate durante la navigazione.
		\begin{itemize} \item Versione utilizzata 2.5.2\end{itemize}
\end{document}