\documentclass[../DocumentazioneDelloStudio.tex]{subfiles}

\begin{document}
\section{Soluzione proposta}
	La soluzione proposta prevede quindi di calcolare un percorso all'interno del grafo che rappresenta l'edificio. Essendo gli archi di tale grafo pesati è possibile calcolare tali pesi dinamicamente sulla base delle preferenze espresse da un utente. Ciò viene fatto semplicemente sommando una costante al peso degli archi che l'utente non vorrebbe incontrare nel suo percorso. Il peso degli archi del grafo dell'edificio dipende fortemente dalla natura dell'arco stesso. Per archi normali è rappresentato dalla sua lunghezza (approssimativa) in metri, mentre per scale e ascensori è regolato da funzioni dipendenti dal numero di piani che l'arco attraversa.\\
	Il calcolo di tali pesi sfrutta due funzioni esponenziali simmetriche rispetto l'asse y. L'idea è che al crescere dei piani il peso delle scale aumenti, mentre quello dell'ascensore diminuisca. \\
	fe(x) = e^(x-k)}\\
	fs(x) = e^(-(x-k))}\\
	Dove \textit{fe(x)} rappresenta la funzione per gli ascensori, \textit{fs(x)} per le scale, x è la differenza di piani che l'arco attraversa e dove k è una costante tendente al numero di piani per cui prendere le scale e gli ascensori hanno lo stesso peso. 
	Noi abbiamo preso un numero di poco inferiore a 2 cioè k = 1.9999. Ciò vuol dire che per noi il numero di piani da attraversare prendendo scale o ascensori è 2, con preferenza per l'ascensore per arrivare al secondo piano.
	Sono state scelte tali funzioni per avere archi tutti con peso positivo ed avere la possibilità di utilizzare Dijktra.
\end{document}